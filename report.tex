%%%
%%% HEADER
%%%

\documentclass[a4paper]{article}

\usepackage[utf8]{inputenc}
\usepackage[T1]{fontenc}
\usepackage[czech]{babel}
\usepackage{listings} 
\usepackage[pdfborder={1 0 0}]{hyperref} 
\usepackage{graphicx}
\usepackage{wrapfig}
\usepackage{subfig}
\usepackage{color}
\usepackage{xcolor}
\usepackage{amsmath}
\usepackage{caption}
\usepackage[a4paper,vmargin={35mm, 35mm}, hmargin={35mm, 35mm}]{geometry}
\usepackage[affil-it]{authblk} 
\usepackage{etoolbox}
\usepackage{lmodern}
\usepackage{titlesec}

%% colors
\definecolor{titleblue}{rgb}{0.02, 0.37, 0.55}
\definecolor{sectionblue}{rgb}{0.25, 0.44, 0.60}
\definecolor{subsectionblue}{rgb}{0.37, 0.59, 0.77}
\definecolor{gray}{rgb}{0.9,0.9,0.9}

%% mod title section
\makeatletter
% line hackity hack
\let\old@rule\@rule
\def\@rule[#1]#2#3{\textcolor{titleblue}{\old@rule[#1]{#2}{#3}}}
\renewcommand{\@maketitle}{
  % font family shortcuts ref: http://tex.stackexchange.com/a/25251
  \begin{flushleft}
  {\color{titleblue}
   \fontfamily{lmss}
   \fontsize{26}{0}
   \selectfont
   \@title
   }{}{}
  \end{flushleft}

  \noindent\makebox[\linewidth]{\rule{\textwidth}{0.1pt}}
  \par
  {\fontsize{1}{0}\@author} %authorhack
  \fontfamily{lmss}\fontsize{12}{0}\color{subsectionblue}\\
  
  \noindent\textit{Datum: \@date}
}
\makeatother

%% mod styles
\titleformat*{\section}{\LARGE\bfseries\sffamily\color{sectionblue}}
\titleformat*{\subsection}{\Large\bfseries\sffamily\color{subsectionblue}}
\titleformat*{\subsubsection}{\large\bfseries\sffamily\color{subsectionblue}}

%% mod listings
\DeclareCaptionFont{white}{\color{white}}
\DeclareCaptionFont{black}{\color{black}}
\DeclareCaptionFormat{listing}{\colorbox{gray}{\parbox{\textwidth}{#1#2#3}}}
\captionsetup[lstlisting]{format=listing,labelfont=black,textfont=black}


%%%
%%% DOCUMENT
%%%

\begin{document}

\title{IIB - Report z Big File Upload flow}
\author{-} % authorhack
\date{\today}
\maketitle

\section{Úvod}

Report obsahuje poznatky z testovania zjednodušeného flow pre upload veľkých súborov (rádovo desiatky MB).
Odoslanie súboru prebieha pomocou $curl$ POST requestu.

\section{Testy}

\subsection{Testovacie súbory}

Súbory použité k testovaniu:
\begin{itemize}
	\item BigFileUpload.msgflow - súčasť súboru project interchange
	\item TestFile1.pdf - 10MB súbor
	\item TestFile2.pdf - 20MB súbor
\end{itemize}

\subsection{Curl requesty}

Testovanie prebehlo na systéme Win7. Data prijatého súboru boli uložené do c:/temp/uploaded-file a header requestu do c:/temp/HTTP-input-header.

\lstset{title=Vstup, language=bash}
\begin{lstlisting}
$ curl --request POST "http://localhost:7800/upload" --data "test"
\end{lstlisting}

Tento request funguje bez problémov a správne uloží data a header.

\lstset{title=Vstup, language=bash}
\begin{lstlisting}
$ curl --request POST "http://localhost:7800/upload" 
	   --data-binary @TestFile1.pdf
\end{lstlisting}

Tento request nefunguje. Po jeho vykonaní sa integračný server ako keby zasekne a nejde s ním chvíľu pracovať.
Často to je potrebné riešiť reštartovaním IBM Integration Toolkit a následne čakať kým sa integračný server spametá.

\lstset{title=Vstup, language=bash}
\begin{lstlisting}
$ curl --request POST "http://localhost:7800/upload" 
	--data-binary @TestFile1.pdf 
	--header "Content-type:multipart/form-data"
$ curl --request POST "http://localhost:7800/upload" 
	--data-binary @TestFile2.pdf 
	--header "Content-type:multipart/form-data"
\end{lstlisting}

Tento request funguje bez problémov a správne data a header uloží iba ak nieje zapnutý debug mode. Tento problém je možno
spôsobený nastavením JVM. Je potrebné zdvyhnúť alocated memory pre veľké súbory.

\subsection{User-defined header}

Použitie user-defined headeru je v IIB jednoduché. Pri každom requeste sa celý header uloží do prijatej message.
\lstset{title=ESQL príklad, language=sql}
\begin{lstlisting}
SET Environment = InputRoot.HTTPInputHeader;
\end{lstlisting}
\section{Záver}

Upload dat pomocou curl POST requestu funguje v poriadku iba pri menších súboroch a pri non-binary datach. Akonáhle sú posielané 
data ako data-binary je potrebné správne nastaviť v headery Content-type inak to spôsobí chvíľkové vytuhnutie integračného serveru.

Pri veľkých súboroch a potrebe použitia debug módu je potrebné zmeniť alocated memory v JVM.



\end{document}